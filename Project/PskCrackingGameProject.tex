\documentclass[12pt,a4paper]{report}
\usepackage[utf8]{inputenc}
\usepackage{graphicx}
\usepackage{xcolor}
\usepackage{listings}
\usepackage{longtable}
\usepackage{geometry}
\usepackage{fancyhdr}
\usepackage{hyperref}
\usepackage{amsmath}
\usepackage{lscape}
\usepackage{float}
\usepackage{array}

\definecolor{codegreen}{rgb}{0,0.6,0}
\definecolor{codegray}{rgb}{0.5,0.5,0.5}
\definecolor{codepurple}{rgb}{0.58,0,0.82}
\definecolor{backcolour}{rgb}{0.95,0.95,0.92}

\lstdefinestyle{mystyle}{
    backgroundcolor=\color{backcolour},   
    commentstyle=\color{codegreen},
    keywordstyle=\color{magenta},
    numberstyle=\tiny\color{codegray},
    stringstyle=\color{codepurple},
    basicstyle=\ttfamily\footnotesize,
    breakatwhitespace=false,         
    breaklines=true,                 
    captionpos=b,                    
    keepspaces=true,                 
    numbers=left,                    
    numbersep=5pt,                  
    showspaces=false,                
    showstringspaces=false,
    showtabs=false,                  
    tabsize=2
}

\lstset{style=mystyle}

\geometry{
    a4paper,
    left=25mm,
    right=25mm,
    top=25mm,
    bottom=25mm
}

\pagestyle{fancy}
\fancyhf{}
\rhead{PSK Cracking Game}
\lhead{HITEC University - Cybersecurity}
\cfoot{\thepage}

\begin{document}

\begin{titlepage}
    \centering
    \vspace*{1cm}
    \includegraphics[width=0.3\textwidth]{hitec_logo.png}\\
    \vspace{1cm}
    {\Large \textbf{HITEC UNIVERSITY TAXILA}}\\
    \vspace{0.5cm}
    {\large \textbf{DEPARTMENT OF CYBERSECURITY}}\\
    \vspace{0.5cm}
    \rule{\textwidth}{0.5mm}\\
    \vspace{0.5cm}
    {\Huge \textbf{PSK CRACKING GAME}}\\
    \vspace{0.5cm}
    {\Large A Password Security Demonstration Tool}\\
    \vspace{0.5cm}
    \rule{\textwidth}{0.5mm}\\
    \vspace{1.5cm}
    
    \begin{tabular}{>{\bfseries}l l}
    Submitted By: & M. Ahad Shaheen (24-CYS-053)\\
    & M. Faizan Muzaffar (24-CYS-037)\\
    \end{tabular}
    \vspace{1cm}
    
    \begin{tabular}{>{\bfseries}l l}
    Submitted To: & Mr. Muhammad Khalid\\
    Course: & CS-207 Computer Organization \& Assembly Language\\
    Semester: & 2nd Semester 2024\\
    \end{tabular}
    \vspace{1cm}
    
    \textbf{Date of Submission:}\\
    \vspace{0.2cm}
    \today\\
    \vfill
\end{titlepage}

\chapter*{Declaration}
We declare that this project report titled \textbf{"PSK Cracking Game"} is our original work and has not been submitted elsewhere for any other degree or qualification.

\vspace{1cm}
\begin{tabular}{ll}
\hspace{5cm} & \hspace{5cm} \\
\cline{1-1} \cline{2-2}
M. Ahad Shaheen & M. Faizan Muzaffar \\
24-CYS-053 & 24-CYS-037 \\
\end{tabular}

\tableofcontents
\newpage

\chapter*{Abstract}
This report documents the development of PSK Cracking Game, a Java application developed for HITEC University's Cybersecurity program. The project demonstrates brute-force password cracking techniques through a graphical user interface, serving as an educational tool for understanding password security vulnerabilities. The application features password generation, cracking simulation, and session logging capabilities.

\chapter{Introduction}
\section{Project Background}
The PSK Cracking Game was developed as part of the 2nd semester curriculum in HITEC University's Cybersecurity program. This project:

\begin{itemize}
    \item Demonstrates practical password cracking techniques
    \item Implements core cybersecurity concepts in Java
    \item Serves as an educational tool for security awareness
\end{itemize}

\section{Team Composition}
The project was developed by:
\begin{itemize}
    \item \textbf{M. Ahad Shaheen (24-CYS-053)}: GUI development and core algorithm implementation
    \item \textbf{M. Faizan Muzaffar (24-CYS-037)}: File I/O operations and testing
\end{itemize}

\chapter{System Design}
\section{Technical Specifications}
\begin{table}[H]
    \centering
    \begin{tabular}{|l|l|}
    \hline
    \textbf{Component} & \textbf{Specification} \\ \hline
    Programming Language & Java 11 \\ \hline
    GUI Framework & Java Swing \\ \hline
    Development Environment & IntelliJ IDEA \\ \hline
    Target Platform & Windows/Linux \\ \hline
    \end{tabular}
    \caption{Technical Specifications}
\end{table}

\chapter{Implementation}
\section{Core Features}
\begin{itemize}
    \item Password generation and storage
    \item Brute-force cracking algorithm
    \item Session logging
    \item User-friendly GUI
\end{itemize}

\section{Code Samples}
\begin{lstlisting}[language=Java,caption=Password Generation]
public void setInputPsk(String inputPsk) {
    this.inputPsk = inputPsk;
}
\end{lstlisting}

\begin{lstlisting}[language=Java,caption=Cracking Algorithm]
private void tryAll(String guess, String target, int maxLength) {
    if (found || guess.length() > maxLength) return;
    if (guess.equals(target)) {
        found = true;
        CrackedPsk = guess;
        return;
    }
    for (int i = 0; i < CHARSET.length(); i++) {
        if (found) return;
        attempts++;
        tryAll(guess + CHARSET.charAt(i), target, maxLength);
    }
}
\end{lstlisting}

\chapter{Results}
\section{Testing Outcomes}
\begin{table}[H]
    \centering
    \begin{tabular}{|c|c|c|c|}
    \hline
    Test Case & Input & Expected Result & Actual Result \\ \hline
    Password Generation & "Secure123" & Stored correctly & Passed \\ \hline
    Short Password Crack & "Hi" & <1 second & Passed \\ \hline
    File Operations & N/A & Proper save/load & Passed \\ \hline
    \end{tabular}
    \caption{Test Results}
\end{table}

\section{Screenshots}
\begin{figure}[H]
    \centering
    \includegraphics[width=0.7\textwidth]{gui_screenshot.png}
    \caption{Application Interface}
\end{figure}

\chapter{Conclusion}
\section{Key Achievements}
\begin{itemize}
    \item Successful implementation of brute-force algorithm
    \item Functional GUI with all specified features
    \item Proper file handling for session persistence
\end{itemize}

\section{Future Enhancements}
\begin{itemize}
    \item Implement dictionary attacks
    \item Add password strength meter
    \item Multi-threading for performance
\end{itemize}

\appendix
\chapter{Source Code}
The complete source code is available at:\\
\url{https://github.com/MuhammadAhadShaheen/PskCrackingGame.git}

\chapter*{References}
\begin{itemize}
    \item Oracle Java Documentation
    \item HITEC University Cybersecurity curriculum
    \item "Java Swing" by TutorialsPoint
\end{itemize}

\end{document}
